\documentclass[12pt,a4paper]{abntex2}
\usepackage[brazil]{babel}
\usepackage{arydshln}
\usepackage{graphicx}
%\setlength{\textwidth}{16cm}

\title{Laboratio de exercicios 0}
\author{Luis Eduardo Castro Mendes de Lima}

\begin{document}

\maketitle

\section{Programação Estruturada}

A programação estruturada é um tipo de programação que geralmente converte programas grandes ou
complexos em pedaços de código pequenos e mais gerenciáveis.

Esses pequenos códigos são geralmente conhecidos como funções ou módulos ou subprogramas de
grandes programas complexos.

É conhecido como programação modular e minimiza as chances de uma função afetar outra função.


\section{Programação não Estruturada}

 A programação não estruturada é um tipo de programação que geralmente executa em ordem
 sequencial, ou seja, esses programas simplesmente não saltam de nenhuma linha de código e
 cada linha é executada sequencialmente.

 É também conhecida como programação não estruturada, capaz de criar algoritmos completos de
 torneamento.\\

\begin{table}%[htbp]
  %\resizebox{\textwidth}{!}{%
    \begin{tabular}{|c|} \hline \\
    Programação Estruturada \\\\ \hline \\
    É basicamente um subconjunto de programas procedimentais. \\ \\ \hdashline \\
    Nesse caso, os programadores podem codificar um programa\\
    simplesmente dividindo-o em módulos ou unidades menores.\\ \\ \hdashline \\
    Suas vantagens incluem reduzir a complexidade,\\
    facilitar a depuração, aumentar a produtividade dos\\
    programas do programador, etc. \\ \\ \hdashline \\
    Esses programas não permitem a duplicação de código.\\ \\ \hdashline \\
    Ele não oferece total liberdade aos programadores para\\
    programarem como quiserem. \\ \\ \hline
    \end{tabular}
 % }
\end{table}

\begin{table}%[htbp]
% \resizebox{\textwidth}{!}{%
    \begin{tabular}{|c|} \hline \\
    Programação não Estruturada \\\\ \hline \\
    É basicamente um programa de procedimentos. \\ \\\hdashline \\
    Nesse caso, os programadores não têm permissão para\\
    dividir o código dos programas em unidades pequenas.\\
    Em vez disso, o programa deve ser escrito como um\\
    único bloco contínuo sem nenhuma quebra. \\ \\\hdashline \\
    Suas vantagens incluem sua velocidade. \\ \\\hdashline \\
    Esses programas permitem a duplicação de código.  \\ \\\hdashline \\
    Ele fornece total liberdade aos programadores \\
    para programar como quiserem. \\ \\\hline

    \end{tabular}
 % }
\end{table}

\end{document}

